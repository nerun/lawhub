% PREÂMBULO =======================================

% Papel A4, fonte tamanho 12, paginação automática
\documentclass[a4paper, 12pt]{article}

% Margens
\usepackage[top=2cm, bottom=2cm, left=2cm, right=2cm]{geometry}

% Acentuação gráfica "ç", "ã", "é" etc.
\usepackage[utf8]{inputenc}

% Idioma português do Brasil
% O pacote Babel apresentado na introdução permite o uso de caracteres especiais e também traduz alguns elementos do documento. Este pacote também ativa automaticamente as regras de hifenização apropriadas para o idioma que você escolher.
\usepackage[brazilian]{babel}

% Indenta o primeiro parágrafo após uma seção, do contrário fica sem.
\usepackage{indentfirst}

% Permite adicionar hyperlinks.
\usepackage{hyperref}
\hypersetup{
    colorlinks=true,
    linkcolor=black,	% Sumário fica preto
    filecolor=magenta,
    urlcolor=blue,		% Link fica azul
}

% Underline avançado.
% Quando a frase tem letras que ficam abaixo da linha, como a letra "g", o underline fica muito afastado. Isto permite usar o smash:
% \underline{\smash{Legal Information Institute}}}
\usepackage{soul}

% Remove a numeração das seções numeradas. Às vezes você precisa de uma seção não numerada e usa o \section*{•}, só que ele não aparece no sumário. Aí você teria que adicionar seção por seção com o \addcontentsline{toc}{section}{nome da seção}. Ficaria grande demais. Aqui basta essa linha e tudo se resolve.
\setcounter{secnumdepth}{0}

% Pré-configuração de listas (mais compactas)
\usepackage{enumitem}
\setlist[enumerate]{itemsep=1pt, parsep=1pt, topsep=1pt}
\setlist[itemize]{itemsep=1pt, parsep=1pt, topsep=1pt}

% Pacotes básicos de matemática pra usar \dfrac
% AMSMATH - AMSFONTS - AMSSYMB
% Equações - Símbolos matemáticos - Caracteres especiais
\usepackage{amsmath, amsfonts, amssymb}

% Lorem ipsum
% usar \lipsum[1] para invocar apenas o 1º parágrafo
\usepackage{lipsum}

% CABEÇALHO
\title{\textbf{Código dos Estados Unidos}}
\author{\textit{tradução de}\\Daniel Dias Rodrigues\\ (\textit{27 de Setembro de 2021})}
\date{}

% CORPO DO DOCUMENTO ==============================

\begin{document}

% Insere o cabeçalho definido no preâmbulo
\maketitle

% Sumário
\tableofcontents

\begin{center}
Exemplo de Centralização
\end{center}

% alinhar à direita
\begin{flushright}
\textbf{Alin}\textit{har} à di\textbf{rei}ta
\end{flushright}

\underline{underline}

\begin{center}
\rule{7cm}{0.4pt}
\end{center}

\section{Introdução}

A ideia central do \LaTeX\ é distanciar o autor o máximo possível da apresentação visual da informação.

Ao invés de trabalhar com ideias visuais, o usuário é encorajado a trabalhar com conceitos mais lógicos --- e, consequentemente, independente da apresentação --- como capítulos, seções, ênfase e tabelas, sem contudo impedir o usuário da liberdade de indicar, expressamente, declarações de formatação.

% Isto é um comentário que não será processado. Ele serve apenas para fazer anotações não incluídas no resultado final. Atenção ao símbolo do comentário: porcentagem (%).

\subsection{Quebra de linha}

Use barras duplas\\ pra quebrar uma linha.

\subsection{Listas}

% Os TABs não são necessários, é só pra facilitar a leitura do código.
Exemplo de lista \textbf{numerada}:

% Padrão: [left=5mm..10mm] a bolinha começa a 0,5cm da margem, e o texto a 1,0 cm da margem
% [left=10mm..15mm] aqui e não no preâmbulo, porque só o primeiro nível deve começar com espaçamento maior da margem, se colocar no cabeçalho todos os níveis ficam muito longe.

\begin{enumerate}[left=15mm..20mm]
	\item Lorem ipsum dolor sit amet
	\begin{enumerate}
		\item Lorem ipsum dolor sit amet
		\begin{enumerate}
			\item Lorem ipsum dolor sit amet
		\end{enumerate}
	\end{enumerate}
\end{enumerate}

Exemplo de lista \textbf{\underline{não} numerada}:

\begin{itemize}[left=15mm..20mm]
	\item Lorem ipsum dolor sit amet
	\begin{itemize}
		\item Lorem ipsum dolor sit amet
		\begin{itemize}
			\item Lorem ipsum dolor sit amet
		\end{itemize}
	\end{itemize}
\end{itemize}

\subsection{Operações básicas e equações}

\begin{itemize}[left=15mm..20mm]
	\item adição: $a+b$
	\item subtração: $a-b$
	\item multiplicação: $a \cdot b$ ou $a \times b$
	\item divisão: $a\div b$; ou fração esprimida pra caber na linha $\frac{a}{b}$
	\item ou uma fração não esprimida na linha: $\dfrac{a}{b}$
	\item potência: $a^{b+c}$; e $a^{10}$ nunca $a^10$
	\item índice: $a_{10}$ nunca $a_10$
\end{itemize}

Equação centralizada, fórmula de Bhaskara: $$\frac{-b \pm \sqrt{b^2 - 4ac}}{2a}$$

\verb|\lipsum[1]|

\vspace{5mm}

\lipsum[1]

\section{Outros}

\begin{enumerate}[label=(\arabic*), left=15mm..22mm]
	\item Copyright: \textbackslash{textcopyright} \textcopyright \space e \textbackslash{copyright} \copyright
	\item Registered: \textbackslash{textregistered} \textregistered
	\item Aspas \verb|``aqui"| - pra poder fechar as ``aspas direitos", do "contrário"
	\item Pula uma linha, como acima: \verb|\{vspace}{5mm}|
	\item mailto: \verb|\href{url}{\underline{\smash{link}}}|
\end{enumerate}

\end{document}
