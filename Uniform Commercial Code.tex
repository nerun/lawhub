% -----------------------------------------------------------------------------
% PREÂMBULO ~ [LuaLaTeX]
% -----------------------------------------------------------------------------
\documentclass[a4paper, 12pt]{book}             % Tipo de documento:
                                                %   [papel, tamanho da fonte]
                                                %   {classe}

\usepackage[
    top=2cm,
    bottom=2cm,
    left=2cm,
    right=2cm
]{geometry}                                     % Define as margens da página

\usepackage[brazilian]{babel}                   % Hifenização e tradução em
                                                % português do Brasil

\usepackage{fontspec}                           % Seleção de fontes OpenType
\usepackage{mathtools}                          % Fórmulas matemáticas e
                                                % símbolos. Carrega amsmath,
                                                % corrige bugs antigos dele e
                                                % adiciona comandos úteis
\usepackage{unicode-math}                       % Matemática em Unicode
\setmainfont{TeX Gyre Termes}                   % Fonte Times-like para o texto
\setmathfont{TeX Gyre Termes Math}              % Fonte matemática compatível

\usepackage{microtype}                          % Melhora a tipografia com
                                                % microajustes

\usepackage{setspace}                           % Permite configurar
                                                % espaçamento entre linhas
\singlespacing                                  % Define espaçamento simples
                                                % entre linhas
\usepackage{indentfirst}                        % Indenta o primeiro parágrafo
                                                % de cada seção
\usepackage{titlesec}                           % Permite customizar títulos
                                                % de seções

\titleformat{\paragraph}                        % Define o estilo do título
                                                % de parágrafos
    {\normalfont\normalsize\bfseries}           % Texto normal, tamanho normal,
                                                % negrito
    {\theparagraph}                             % Exibe o número do parágrafo
    {1em}                                       % Espaço entre o número e
                                                % o título
    {}

\titlespacing*{\paragraph}                      % Define o espaçamento do
                                                % título de parágrafo
    {0pt}
    {1.5ex plus 1ex minus .2ex}
    {1ex plus .2ex}                             % Espaçamentos antes e depois
                                                % do parágrafo

\titleformat{\subparagraph}                     % Define o estilo do título
                                                % de subparágrafos
    {\normalfont\normalsize\bfseries\itshape}   % Texto normal, tamanho normal,
                                                % negrito e itálico
    {\thesubparagraph}                          % Exibe o número do
                                                % subparágrafo
    {1em}                                       % Espaço entre o número e
                                                % o título
    {}

\titlespacing*{\subparagraph}                   % Define o espaçamento do
                                                % título de subparágrafo
    {0pt}
    {1ex plus 0.5ex minus .2ex}
    {0.5ex plus .2ex}                           % Espaçamentos antes e depois
                                                % do subparágrafo

\addto\captionsbrazilian{                       % Centraliza e renomeia a seção
                                                % de sumário
    \renewcommand{\contentsname}{%
        \centering Sumário%
    }
}

\setcounter{secnumdepth}{-2}                    % Define o nível máximo de
                                                % numeração de seções
                                                %  0 → numera só \part
                                                %  1 → numera \chapter
                                                %  2 → numera \section
                                                %  3 → numera \subsection
                                                % -2 → nada é numerado

\setlength{\skip\footins}{0.75cm}               % Espaçamento entre o texto e
                                                % as notas de rodapé
\usepackage[hang]{footmisc}                     % Estilo de nota com recuo
                                                % ("hang" = pendente)
\setlength{\footnotemargin}{3mm}                % Define margem esquerda para
                                                % notas de rodapé

\usepackage{enumitem}                           % Personaliza listas
                                                % (itemize/enumerate)
\setlist[enumerate]{                            % Espaçamentos em listas
    itemsep=1pt,
    parsep=1pt,
    topsep=1pt
}
\setlist[itemize]{                              % Espaçamentos em listas não
    itemsep=1pt,                                % numeradas
    parsep=1pt,
    topsep=1pt
}

\usepackage[svgnames, x11names]{xcolor}         % Permite usar nomes SVG e X11
                                                % para cores
\usepackage{lipsum}                             % Gera texto de exemplo
                                                % (Lorem Ipsum)
\SetLipsumLanguage{brazilian}                   % Evita que lipsum tente mudar
                                                % a língua para latim
\usepackage{soul}                               % Permite sublinhar, riscar e
                                                % destacar texto

\newcommand{\autor}{Daniel Dias Rodrigues}      % Define comando personalizado
                                                % para o autor
\newcommand{\titulo}{Código Comercial Uniforme} % Define comando personalizado
                                                % para o título

\title{\titulo}                                 % Define o título. Concatena
\author{\textit{tradução de}\\\autor}           % Define o autor, usando o
                                                % comando personalizado

\usepackage[useregional]{datetime2}             % Personaliza o formato da data
\date{\DTMtoday}                                % A data do documento é "hoje"

\newcommand{\us}[1]{                            % Define comando que sublinha
    \underline{\smash{#1}}                      % sem afetar o espaçamento
}                                               % vertical

\usepackage[
    pdfauthor={\autor},                         % Define o autor nos metadados
                                                % do PDF
    pdftitle={\titulo},                         % Define o título nos metadados
                                                % do PDF
    pdfsubject={Tradução de lei americana},     % Campo de assunto do PDF
    pdfkeywords={
        lei,
        direito,
        estados unidos,
        tradução}                               % Define palavras-chave do PDF
]{hyperref}                                     % Ativa links no documento

\hypersetup{
    colorlinks=true,                            % Habilita links coloridos
    linkcolor=black,                            % Cor dos links internos
    filecolor=magenta,                          % Cor de links para arquivos
    urlcolor=blue                               % Cor de links para URLs
}

% -----------------------------------------------------------------------------
% CORPO DO DOCUMENTO
% -----------------------------------------------------------------------------
\begin{document}

\maketitle

\textit{Copyright \textcopyright 1978, 1987, 1988, 1990, 1991, 1992, 1994, 1995, 1998, 2001, 2004, 2010, 2011, 2012 pelo ``American Law Institute'' e pela ``National Conference of Commissioners on Uniform State Laws''; reproduzido, publicado e distribuído com a permissão do Conselho Editorial Permanente do Código Comercial Uniforme para os fins limitados de estudo, ensino e pesquisa acadêmica.}

Nossa coleção tem como objetivo disponibilizar o C.C.U. na versão mais amplamente adotada pelos estados. \textbf{Isso significa que nem sempre usaremos a revisão mais recente se essa revisão não tiver alcançado ampla adoção entre as legislaturas americanas.}

Devido a restrições de licença, esta versão do C.C.U. não inclui os comentários oficiais.

\begin{center}\rule{7cm}{0.4pt}\end{center}

Esta é uma tradução do \href{https://www.law.cornell.edu/ucc}{\underline{Uniform Commercial Code}} conforme publicado pelo \textit{Legal Information Institute} da Universidade Cornell.

\textit{O texto desta tradução é copyright \textcopyright 2021 de Daniel Dias Rodrigues. Alguns direitos reservados. Os direitos autorais sobre traduções são protegidos pela Convenção de Berna art. 2, alínea 3 (Decreto nº 75.699/75) e pela Lei de Direitos Autorais (``LDA'') art. 7º caput e inciso XI (Lei Federal nº 9.610/98). Esta tradução é distribuída sob os termos da licença \href{https://creativecommons.org/licenses/by/4.0/deed.pt_BR}{\underline{\smash{Creative Commons BY 4.0}}}. Por essa licença tudo é permitido, inclusive alterar a tradução e lucrar com ela sem ter que pagar royalties, desde que você cite o nome do tradutor. Também pela LDA art. 53, inciso II, além do título original, você também é obrigado a citar o nome do tradutor. O tradutor é titular de direitos autorais (LDA art. 14). Isso é um direito moral do autor (LDA art. 24, II). Os direitos morais do autor são inalienáveis e irrenunciáveis (LDA art. 27).}

Para erros de tradução: \href{mailto:danieldiasr@gmail.com}{\underline{\smash{danieldiasr@gmail.com}}}.

\tableofcontents

\part[Artigo 2 -- VENDAS (2002)]{Artigo 2\\ VENDAS (2002)}

\chapter[Parte 3 -- OBRIGAÇÃO GERAL E FORMAÇÃO DO CONTRATO]{Parte 3\\ OBRIGAÇÃO GERAL E FORMAÇÃO DO CONTRATO}

\section{§ 2-315. Garantia implícita: adequação a um determinado fim.}

Quando o vendedor, no momento da contratação, tiver conhecimento de qualquer propósito em particular para o qual o produto é necessário e de que o comprador está contando com a habilidade ou julgamento do vendedor para selecionar ou fornecer o produto adequado, então haverá, a não ser que seja excluída ou modificada na próxima seção, uma garantia implícita de que os produtos serão adequados para esse fim.

\section{§ 2-316. Exclusão ou modificação de garantias.}

\begin{enumerate}[label=(\arabic*)]
	\item Palavras ou conduta relevantes para a criação de uma garantia expressa e palavras ou conduta tendentes a negar ou limitar a garantia, devem ser interpretadas sempre que possível como consistentes umas com as outras; mas, sujeito às disposições deste Artigo sobre liberdade condicional ou evidência extrínseca (Seção 2-202), a negação ou limitação não terá efeito na medida em que tal interpretação não seja razoável.
	\item Sujeito à subseção (3), para excluir ou modificar a garantia implícita de comerciabilidade ou qualquer parte dela, o texto deve mencionar a comerciabilidade de forma conspícua, e para excluir ou modificar qualquer garantia implícita de adequação, a exclusão deve ser por escrito e conspícua. A linguagem para excluir todas as garantias implícitas de adequação será suficiente se ela declarar, por exemplo, que ``Não há garantias que se estendam para além daquelas aqui apresentadas''.
	\item Não obstante a subseção (2):
	\begin{enumerate}
		\item a menos que as circunstâncias indiquem o contrário, todas as garantias implícitas são excluídas por expressões como ``no estado em que se encontra'', ``com todas as falhas'' ou outra redação que, no entendimento comum, chama a atenção do comprador para a exclusão de garantias e deixa claro que não há nenhuma garantia implícita; e
		\item quando o comprador, antes de celebrar o contrato, tiver examinado o produto, a amostra ou o modelo tão completamente quanto ele desejava, ou tiver se recusado a examinar o produto, não haverá garantia implícita no que diz respeito aos defeitos que um exame deveria, nessas circunstâncias, ter revelado para ele; e
		\item uma garantia implícita também pode ser excluída ou modificada pelo curso da negociação, pelo curso do desempenho ou pelos usos e costumes do comércio.
	\end{enumerate}
	\item As indenizações por violação de garantia podem ser limitadas conforme as disposições deste Artigo sobre liquidação ou limitação de danos e sobre modificação contratual das indenizações (Seções 2-718 e 2-719).
\end{enumerate}

\chapter[Parte 7 -- MEDIDAS DE REPARAÇÃO]{Parte 7\\ MEDIDAS DE REPARAÇÃO}

\section{§ 2-711. Medidas de Reparação do Comprador em Geral; Garantia Real do Comprador sobre Mercadorias Rejeitadas.}

\begin{enumerate}[label=(\arabic*)]
	\item Quando o vendedor não fizer a entrega ou repudiar, ou o comprador rejeitar legitimamente ou revogar justificadamente a aceitação então, em relação a quaisquer mercadorias envolvidas, e em relação ao todo, se a violação abranger todo o contrato (Seção 2-612), o comprador poderá cancelar e, quer o tenha feito ou não, poderá, além de recuperar a parte do preço que foi pago:
	\begin{enumerate}
		\item ``cobrir'' e obter reparação de danos nos termos da próxima seção quanto a todos os bens afetados, tenham ou não sido identificados no contrato; ou
		\item obter reparação de danos por não entrega conforme previsto neste Artigo (Seção 2-713).
	\end{enumerate}
	\item Quando o vendedor não entregar ou repudiar, o comprador também poderá:
	\begin{enumerate}
		\item se as mercadorias tiverem sido identificadas, recuperá-las conforme previsto neste Artigo (Seção 2-502); ou
		\item em um caso adequado, obter desempenho específico ou reabastecer as mercadorias conforme previsto neste Artigo (Seção 2-716).
	\end{enumerate}
	\item Na rejeição legítima ou revogação justificável da aceitação, um comprador tem uma garantia real sobre os bens em sua posse ou controle por quaisquer pagamentos feitos sobre seu preço e quaisquer despesas razoavelmente incorridas em sua inspeção, recebimento, transporte, cuidado e custódia e pode reter tais bens e revendê-los da mesma maneira que um vendedor lesado (Seção 2-706).
\end{enumerate}

\section{§ 2-712. ``Cobertura''; Aquisição de bens substitutos pelo comprador.}

\begin{enumerate}[label=(\arabic*)]
	\item Após uma violação da seção anterior, o comprador poderá ``cobrir'', fazendo de boa fé e sem demora injustificada, qualquer compra razoável ou contrato de compra de mercadorias em substituição às devidas pelo vendedor.
	\item O comprador poderá recuperar do vendedor, como danos, a diferença entre o custo da cobertura e o preço do contrato, juntamente com quaisquer danos incidentais ou consequenciais, conforme definido a seguir (Seção 2-715), subtraídas as despesas economizadas em consequência da violação do vendedor.
	\item A falha do comprador em efetuar a cobertura dentro desta seção não o impede de obter qualquer outra medida de reparação.
\end{enumerate}

\end{document}
