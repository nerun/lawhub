% PREÂMBULO =======================================

% Papel A4, fonte tamanho 12, paginação automática
\documentclass[a4paper, 12pt]{article}

% Margens
\usepackage[top=2cm, bottom=2cm, left=2cm, right=2cm]{geometry}

\usepackage{setspace}
%\doublespacing
%\onehalfspacing
%\singlespacing

% Acentuação gráfica "ç", "ã", "é" etc.
\usepackage[utf8]{inputenc}

% Idioma português do Brasil
% O pacote Babel apresentado na introdução permite o uso de caracteres especiais e também traduz alguns elementos do documento. Este pacote também ativa automaticamente as regras de hifenização apropriadas para o idioma que você escolher.
\usepackage[brazilian]{babel}

% Indenta o primeiro parágrafo após uma seção, do contrário fica sem.
\usepackage{indentfirst}

% Permite adicionar hyperlinks.
\usepackage{hyperref}
\hypersetup{
    colorlinks=true,
    linkcolor=black,	% Sumário fica preto
    filecolor=magenta,
    urlcolor=blue,		% Link fica azul
}

% Underline avançado.
% Quando a frase tem letras que ficam abaixo da linha, como a letra "g", o underline fica muito afastado. Isto permite usar o smash:
% \underline{\smash{Legal Information Institute}}}
\usepackage{soul}

% Remove a numeração das seções numeradas. Às vezes você precisa de uma seção não numerada e usa o \section*{•}, só que ele não aparece no sumário. Aí você teria que adicionar seção por seção com o \addcontentsline{toc}{section}{nome da seção}. Ficaria grande demais. Aqui basta essa linha e tudo se resolve.
\setcounter{secnumdepth}{0}

% Pré-configuração de listas (mais compactas)
\usepackage{enumitem}
\setlist[enumerate]{itemsep=1pt, parsep=1pt, topsep=1pt}
\setlist[itemize]{itemsep=1pt, parsep=1pt, topsep=1pt}

% Pacotes básicos de matemática pra usar \dfrac
% AMSMATH - AMSFONTS - AMSSYMB
% Equações - Símbolos matemáticos - Caracteres especiais
\usepackage{amsmath, amsfonts, amssymb}

% Lorem ipsum
% usar \lipsum[1] para invocar apenas o 1º parágrafo
\usepackage{lipsum}

% CABEÇALHO
\title{\textbf{Código dos Estados Unidos}}
\author{\textit{tradução de}\\Daniel Dias Rodrigues\\ (\textit{18 de Abril de 2022})}
\date{}

% CORPO DO DOCUMENTO ==============================

\begin{document}

% Insere o cabeçalho definido no preâmbulo
\maketitle

Esta é uma tradução do \href{https://uscode.house.gov}{\underline{United States Code}} conforme publicado pelo \textit{Office of the Law Revision Counsel of the United States House of Representatives} (Escritório do Conselho de Revisão da Lei da Câmara dos Deputados dos Estados Unidos).

\begin{center}
\rule{7cm}{0.4pt}
\end{center}

\textit{O texto desta tradução é copyright \textcopyright 2022 de Daniel Dias Rodrigues. Alguns direitos reservados. Os direitos autorais sobre traduções são protegidos pela Convenção de Berna art. 2, alínea 3 (Decreto nº 75.699/75) e pela Lei de Direitos Autorais (``LDA") art. 7º caput e inciso XI (Lei Federal nº 9.610/98). Esta tradução é distribuída sob os termos da licença \href{https://creativecommons.org/licenses/by/4.0/deed.pt_BR}{\underline{\smash{Creative Commons BY 4.0}}}. Por essa licença tudo é permitido, inclusive alterar a tradução e lucrar com ela sem ter que pagar royalties, desde que você cite o nome do tradutor. Também pela LDA art. 53, inciso II, além do título original, você também é obrigado a citar o nome do tradutor. O tradutor é titular de direitos autorais (LDA art. 14). Isso é um direito moral do autor (LDA art. 24, II). Os direitos morais do autor são inalienáveis e irrenunciáveis (LDA art. 27).}

\vspace{5mm}

Para erros de tradução: \href{mailto:danieldiasr@gmail.com}{\underline{\smash{danieldiasr@gmail.com}}}.

\begin{center}
\rule{7cm}{0.4pt}
\end{center}

\pagebreak

\tableofcontents

\pagebreak

\section{TÍTULO 7 - DIREITOS AUTORAIS (``COPYRIGHTS")}

\subsection{Capítulo 1 - ASSUNTO E ESCOPO DOS DIREITOS AUTORAIS}

\subsubsection{§ 102 - Assunto dos direitos autorais: em geral}

\begin{enumerate}[label=(\alph*)]
	\item A proteção de direitos autorais subsiste, de acordo com este título, em obras originárias fixadas em qualquer meio de expressão tangível, agora conhecido ou desenvolvido posteriormente, a partir do qual possam ser percebidas, reproduzidas ou de outra forma comunicadas, diretamente ou com o auxílio de uma máquina ou dispositivo. Obras originárias incluem as seguintes categorias:
	\begin{enumerate}[label=(\arabic*)]
		\item obras literárias;
		\item obras musicais, incluindo quaisquer palavras acompanhantes;
		\item obras dramáticas, incluindo qualquer música acompanhante;
		\item pantomimas e trabalhos coreográficos;
		\item obras pictóricas, gráficas e escultóricas;
		\item filmes e outras obras audiovisuais;
		\item gravações de som; e
		\item obras arquitetônicas.
	\end{enumerate}
	\item Em nenhum caso a proteção de direitos autorais para uma obra originária se estende a qualquer ideia, procedimento, processo, sistema, método de operação, conceito, princípio ou descoberta, independentemente da forma com que são descritos, explicados, ilustrados, ou incorporados em tal trabalho.
\end{enumerate}

\end{document}
